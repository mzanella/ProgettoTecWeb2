\documentclass[../ProgettoTecWeb2.tex]{subfiles}

\begin{document}
\section{Introduzione}
	\subsection{Periodo dell'analisi di usabilità}
	Il periodo in cui è stata effettuata l'analisi del sito web è maggio-giugno 2016. Il sito successivamente potrebbe aver subito variazioni e quindi alcune parti di tale analisi potrebbero non essere più consistenti.

	\subsection{Scopo del sito web}
	Il sito \href{http://ilovevg.it/}{I Love Videogames} è un sito web che raccoglie notizie e  recensioni di videogiochi sia per computer che per le varie console.

	\subsection{Il nome del sito web}
	Il nome di un sito web ha influenza sulla sua usabilità, come l'URL che permette di arrivarci. Infatti, banalmente, gli utenti per visitare un sito web devono trovare e ricordarsi il nome di un sito web. Il nome \textit{I Love Videogames} è un nome di certo appropriato per il sito web. Difatti fa subito capire il tema che tratta di cui tratta il sito. Non è troppo lungo e non ha ci sono altri siti web molto più famosi con nome simile. Inoltre il suono gradevole del nome è un ulteriore aspetto positivo.

	\subsection{L'URL del sito web}
	L'URL del sito è \url{http:\\www.ilovevg.it}. Tale scelta non è forse la migliore:
	\begin{itemize}
		\item l'URL è un'abbreviazione del nome. Ciò comporta che un utente debba ricordarsi che il nome e l'URL del sito web sono differenti e quindi aumenta lo sforzo computazionale. Questa scelta è stata probabilmente dovuta dal fatto che il dominio \url{www.ilovevideogames.it} è occupato;
		\item è sempre preferibile comprare il dominio ``.com'', poiché è quello più famoso e quindi che si ricorda meglio. Anche in questo caso la scelta può essere stata dettata dal fatto che il dominio risultasse già occupato.
	\end{itemize}
\end{document}
